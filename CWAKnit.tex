\documentclass[]{article}
\usepackage{lmodern}
\usepackage{amssymb,amsmath}
\usepackage{ifxetex,ifluatex}
\usepackage{fixltx2e} % provides \textsubscript
\ifnum 0\ifxetex 1\fi\ifluatex 1\fi=0 % if pdftex
  \usepackage[T1]{fontenc}
  \usepackage[utf8]{inputenc}
\else % if luatex or xelatex
  \ifxetex
    \usepackage{mathspec}
  \else
    \usepackage{fontspec}
  \fi
  \defaultfontfeatures{Ligatures=TeX,Scale=MatchLowercase}
\fi
% use upquote if available, for straight quotes in verbatim environments
\IfFileExists{upquote.sty}{\usepackage{upquote}}{}
% use microtype if available
\IfFileExists{microtype.sty}{%
\usepackage{microtype}
\UseMicrotypeSet[protrusion]{basicmath} % disable protrusion for tt fonts
}{}
\usepackage[margin=1in]{geometry}
\usepackage{hyperref}
\hypersetup{unicode=true,
            pdftitle={Statistical Computing CW A},
            pdfauthor={Matin Mahmood (s1841215)},
            pdfborder={0 0 0},
            breaklinks=true}
\urlstyle{same}  % don't use monospace font for urls
\usepackage{color}
\usepackage{fancyvrb}
\newcommand{\VerbBar}{|}
\newcommand{\VERB}{\Verb[commandchars=\\\{\}]}
\DefineVerbatimEnvironment{Highlighting}{Verbatim}{commandchars=\\\{\}}
% Add ',fontsize=\small' for more characters per line
\usepackage{framed}
\definecolor{shadecolor}{RGB}{248,248,248}
\newenvironment{Shaded}{\begin{snugshade}}{\end{snugshade}}
\newcommand{\KeywordTok}[1]{\textcolor[rgb]{0.13,0.29,0.53}{\textbf{#1}}}
\newcommand{\DataTypeTok}[1]{\textcolor[rgb]{0.13,0.29,0.53}{#1}}
\newcommand{\DecValTok}[1]{\textcolor[rgb]{0.00,0.00,0.81}{#1}}
\newcommand{\BaseNTok}[1]{\textcolor[rgb]{0.00,0.00,0.81}{#1}}
\newcommand{\FloatTok}[1]{\textcolor[rgb]{0.00,0.00,0.81}{#1}}
\newcommand{\ConstantTok}[1]{\textcolor[rgb]{0.00,0.00,0.00}{#1}}
\newcommand{\CharTok}[1]{\textcolor[rgb]{0.31,0.60,0.02}{#1}}
\newcommand{\SpecialCharTok}[1]{\textcolor[rgb]{0.00,0.00,0.00}{#1}}
\newcommand{\StringTok}[1]{\textcolor[rgb]{0.31,0.60,0.02}{#1}}
\newcommand{\VerbatimStringTok}[1]{\textcolor[rgb]{0.31,0.60,0.02}{#1}}
\newcommand{\SpecialStringTok}[1]{\textcolor[rgb]{0.31,0.60,0.02}{#1}}
\newcommand{\ImportTok}[1]{#1}
\newcommand{\CommentTok}[1]{\textcolor[rgb]{0.56,0.35,0.01}{\textit{#1}}}
\newcommand{\DocumentationTok}[1]{\textcolor[rgb]{0.56,0.35,0.01}{\textbf{\textit{#1}}}}
\newcommand{\AnnotationTok}[1]{\textcolor[rgb]{0.56,0.35,0.01}{\textbf{\textit{#1}}}}
\newcommand{\CommentVarTok}[1]{\textcolor[rgb]{0.56,0.35,0.01}{\textbf{\textit{#1}}}}
\newcommand{\OtherTok}[1]{\textcolor[rgb]{0.56,0.35,0.01}{#1}}
\newcommand{\FunctionTok}[1]{\textcolor[rgb]{0.00,0.00,0.00}{#1}}
\newcommand{\VariableTok}[1]{\textcolor[rgb]{0.00,0.00,0.00}{#1}}
\newcommand{\ControlFlowTok}[1]{\textcolor[rgb]{0.13,0.29,0.53}{\textbf{#1}}}
\newcommand{\OperatorTok}[1]{\textcolor[rgb]{0.81,0.36,0.00}{\textbf{#1}}}
\newcommand{\BuiltInTok}[1]{#1}
\newcommand{\ExtensionTok}[1]{#1}
\newcommand{\PreprocessorTok}[1]{\textcolor[rgb]{0.56,0.35,0.01}{\textit{#1}}}
\newcommand{\AttributeTok}[1]{\textcolor[rgb]{0.77,0.63,0.00}{#1}}
\newcommand{\RegionMarkerTok}[1]{#1}
\newcommand{\InformationTok}[1]{\textcolor[rgb]{0.56,0.35,0.01}{\textbf{\textit{#1}}}}
\newcommand{\WarningTok}[1]{\textcolor[rgb]{0.56,0.35,0.01}{\textbf{\textit{#1}}}}
\newcommand{\AlertTok}[1]{\textcolor[rgb]{0.94,0.16,0.16}{#1}}
\newcommand{\ErrorTok}[1]{\textcolor[rgb]{0.64,0.00,0.00}{\textbf{#1}}}
\newcommand{\NormalTok}[1]{#1}
\usepackage{graphicx,grffile}
\makeatletter
\def\maxwidth{\ifdim\Gin@nat@width>\linewidth\linewidth\else\Gin@nat@width\fi}
\def\maxheight{\ifdim\Gin@nat@height>\textheight\textheight\else\Gin@nat@height\fi}
\makeatother
% Scale images if necessary, so that they will not overflow the page
% margins by default, and it is still possible to overwrite the defaults
% using explicit options in \includegraphics[width, height, ...]{}
\setkeys{Gin}{width=\maxwidth,height=\maxheight,keepaspectratio}
\IfFileExists{parskip.sty}{%
\usepackage{parskip}
}{% else
\setlength{\parindent}{0pt}
\setlength{\parskip}{6pt plus 2pt minus 1pt}
}
\setlength{\emergencystretch}{3em}  % prevent overfull lines
\providecommand{\tightlist}{%
  \setlength{\itemsep}{0pt}\setlength{\parskip}{0pt}}
\setcounter{secnumdepth}{0}
% Redefines (sub)paragraphs to behave more like sections
\ifx\paragraph\undefined\else
\let\oldparagraph\paragraph
\renewcommand{\paragraph}[1]{\oldparagraph{#1}\mbox{}}
\fi
\ifx\subparagraph\undefined\else
\let\oldsubparagraph\subparagraph
\renewcommand{\subparagraph}[1]{\oldsubparagraph{#1}\mbox{}}
\fi

%%% Use protect on footnotes to avoid problems with footnotes in titles
\let\rmarkdownfootnote\footnote%
\def\footnote{\protect\rmarkdownfootnote}

%%% Change title format to be more compact
\usepackage{titling}

% Create subtitle command for use in maketitle
\newcommand{\subtitle}[1]{
  \posttitle{
    \begin{center}\large#1\end{center}
    }
}

\setlength{\droptitle}{-2em}

  \title{Statistical Computing CW A}
    \pretitle{\vspace{\droptitle}\centering\huge}
  \posttitle{\par}
    \author{Matin Mahmood (s1841215)}
    \preauthor{\centering\large\emph}
  \postauthor{\par}
      \predate{\centering\large\emph}
  \postdate{\par}
    \date{25 February 2020}


\begin{document}
\maketitle

\begin{Shaded}
\begin{Highlighting}[]
\CommentTok{#Given in Coursework Document}
\KeywordTok{source}\NormalTok{(}\StringTok{"CWA2020code.R"}\NormalTok{)}
\KeywordTok{suppressPackageStartupMessages}\NormalTok{(}\KeywordTok{library}\NormalTok{(tidyverse))}
\KeywordTok{theme_set}\NormalTok{(}\KeywordTok{theme_bw}\NormalTok{())}
\NormalTok{filament <-}\StringTok{ }\KeywordTok{read.csv}\NormalTok{(}\StringTok{"filament.csv"}\NormalTok{, }\DataTypeTok{stringsAsFactors =} \OtherTok{FALSE}\NormalTok{) }
\end{Highlighting}
\end{Shaded}

\section{Task 1: Actual Weight and CAD Weight
Plot}\label{task-1-actual-weight-and-cad-weight-plot}

\begin{Shaded}
\begin{Highlighting}[]
\KeywordTok{ggplot}\NormalTok{(filament) }\OperatorTok{+}
\StringTok{  }\KeywordTok{geom_point}\NormalTok{(}\KeywordTok{aes}\NormalTok{(CAD_Weight, Actual_Weight, }\DataTypeTok{col =}\NormalTok{ Material)) }\OperatorTok{+}
\StringTok{  }\KeywordTok{scale_color_manual}\NormalTok{(}\DataTypeTok{values =} \KeywordTok{c}\NormalTok{(}\StringTok{"Black"}\NormalTok{ =}\StringTok{ "black"}\NormalTok{,}
                                \StringTok{"Red"}\NormalTok{ =}\StringTok{ "red"}\NormalTok{,}
                                \StringTok{"Green"}\NormalTok{ =}\StringTok{ "green"}\NormalTok{,}
                                \StringTok{"Magenta"}\NormalTok{ =}\StringTok{ "magenta"}\NormalTok{,}
                                \StringTok{"Neon pink"}\NormalTok{ =}\StringTok{ "#fca3b7"}\NormalTok{, }\CommentTok{#differentiate from magenta}
                                \StringTok{"Neon blue"}\NormalTok{=}\StringTok{ "#1b03a3"}\NormalTok{)) }\OperatorTok{+}
\StringTok{  }
\StringTok{  }\KeywordTok{labs}\NormalTok{(}\DataTypeTok{subtitle=}\StringTok{"Data: filament.csv"}\NormalTok{, }
       \DataTypeTok{y=}\StringTok{"Actual Weight"}\NormalTok{, }
       \DataTypeTok{x=}\StringTok{"CAD Weight"}\NormalTok{, }
       \DataTypeTok{title=}\StringTok{"Scatter Plot of Actual Weight vs CAD Weight"}\NormalTok{, }
       \DataTypeTok{caption =} \StringTok{"Note: Colour of points correspond to Material Colour"}\NormalTok{)}
\end{Highlighting}
\end{Shaded}

\includegraphics{CWAKnit_files/figure-latex/unnamed-chunk-2-1.pdf}

\begin{center}\rule{0.5\linewidth}{\linethickness}\end{center}

\section{Task 2: Model Estimate
Function}\label{task-2-model-estimate-function}

\begin{Shaded}
\begin{Highlighting}[]
\NormalTok{model_estimate <-}\StringTok{ }\ControlFlowTok{function}\NormalTok{(formulas, data, response)\{}
  
\NormalTok{  z_data=}\KeywordTok{model_Z}\NormalTok{(formulas,data)}
  
\NormalTok{  opt <-}\StringTok{ }\KeywordTok{optim}\NormalTok{(}\KeywordTok{rep}\NormalTok{(}\DecValTok{0}\NormalTok{, }\KeywordTok{ncol}\NormalTok{(z_data[[}\StringTok{"ZE"}\NormalTok{]])), }
               \DataTypeTok{fn =}\NormalTok{ neg_log_lik,}
               \DataTypeTok{Z =}\NormalTok{ z_data, }
               \DataTypeTok{y =}\NormalTok{ data[[response]],}
               \DataTypeTok{method =} \StringTok{"BFGS"}\NormalTok{,}
               \DataTypeTok{control =} \KeywordTok{list}\NormalTok{(}\DataTypeTok{maxit =} \DecValTok{5000}\NormalTok{), }\CommentTok{# Anouncement on Learn, Ensures Convergence}
               \DataTypeTok{hessian =} \OtherTok{TRUE}\NormalTok{)}
  
\NormalTok{  theta <-}\StringTok{ }\NormalTok{opt}\OperatorTok{$}\NormalTok{par}
\NormalTok{  Sigma_theta <-}\StringTok{ }\KeywordTok{solve}\NormalTok{(opt}\OperatorTok{$}\NormalTok{hessian)}
  
  
  \KeywordTok{return}\NormalTok{ (}\KeywordTok{list}\NormalTok{(}\DataTypeTok{theta =}\NormalTok{ theta, }\DataTypeTok{formulas =}\NormalTok{ formulas, }\DataTypeTok{Sigma_theta=}\NormalTok{Sigma_theta))}
  
\NormalTok{  \}}
\end{Highlighting}
\end{Shaded}

\begin{center}\rule{0.5\linewidth}{\linethickness}\end{center}

\section{Task 3: Estimating Model 3}\label{task-3-estimating-model-3}

\begin{Shaded}
\begin{Highlighting}[]
\NormalTok{data_obs <-}\StringTok{ }\NormalTok{filament[filament}\OperatorTok{$}\NormalTok{Class}\OperatorTok{==}\StringTok{"obs"}\NormalTok{,] }\CommentTok{#Extract Observed Data}

\NormalTok{formulas_}\DecValTok{3}\NormalTok{ <-}\StringTok{ }\KeywordTok{list}\NormalTok{(}\DataTypeTok{E =} \OperatorTok{~}\StringTok{ }\DecValTok{1} \OperatorTok{+}\StringTok{ }\NormalTok{CAD_Weight, }\DataTypeTok{V =} \OperatorTok{~}\StringTok{ }\DecValTok{1}\NormalTok{)}

\NormalTok{estimates_}\DecValTok{3}\NormalTok{ <-}\StringTok{ }\KeywordTok{model_estimate}\NormalTok{(formulas_}\DecValTok{3}\NormalTok{, data_obs, }\StringTok{"Actual_Weight"}\NormalTok{)}

\NormalTok{estimates_}\DecValTok{3}
\end{Highlighting}
\end{Shaded}

\begin{verbatim}
## $theta
## [1] 0.3058488 1.0660628 0.6901968
## 
## $formulas
## $formulas$E
## ~1 + CAD_Weight
## 
## $formulas$V
## ~1
## 
## 
## $Sigma_theta
##               [,1]          [,2]          [,3]
## [1,]  9.026507e-02 -1.708832e-03  6.877290e-08
## [2,] -1.708832e-03  4.791775e-05 -1.475095e-09
## [3,]  6.877290e-08 -1.475095e-09  2.941178e-02
\end{verbatim}

\begin{center}\rule{0.5\linewidth}{\linethickness}\end{center}

\section{Task 4: Model 3 Prediction Intervals
Plot}\label{task-4-model-3-prediction-intervals-plot}

\begin{Shaded}
\begin{Highlighting}[]
\NormalTok{data_test <-}\StringTok{ }\NormalTok{filament[filament}\OperatorTok{$}\NormalTok{Class}\OperatorTok{==}\StringTok{"test"}\NormalTok{,] }\CommentTok{#Extract Test Data}

\NormalTok{model3 <-}\StringTok{ }\KeywordTok{model_predict}\NormalTok{(estimates_}\DecValTok{3}\NormalTok{[[}\StringTok{"theta"}\NormalTok{]], estimates_}\DecValTok{3}\NormalTok{[[}\StringTok{"formulas"}\NormalTok{]], estimates_}\DecValTok{3}\NormalTok{[[}\StringTok{"Sigma_theta"}\NormalTok{]],data_test,}\DataTypeTok{type=}\StringTok{"observation"}\NormalTok{)}


\NormalTok{pred_plot_}\DecValTok{4}\NormalTok{ <-}\StringTok{ }\KeywordTok{cbind}\NormalTok{(data_test, model3)}

\KeywordTok{ggplot}\NormalTok{() }\OperatorTok{+}
\StringTok{  }\KeywordTok{geom_ribbon}\NormalTok{(}\DataTypeTok{data =}\NormalTok{ pred_plot_}\DecValTok{4}\NormalTok{,}
              \KeywordTok{aes}\NormalTok{(CAD_Weight, }\DataTypeTok{ymin =}\NormalTok{ lwr, }\DataTypeTok{ymax =}\NormalTok{ upr),}
              \DataTypeTok{alpha =} \FloatTok{0.25}\NormalTok{, }\DataTypeTok{fill =} \StringTok{"blue"}\NormalTok{) }\OperatorTok{+}
\StringTok{  }\KeywordTok{geom_line}\NormalTok{(}\DataTypeTok{data =}\NormalTok{ pred_plot_}\DecValTok{4}\NormalTok{, }\KeywordTok{aes}\NormalTok{(CAD_Weight, mu), }\DataTypeTok{col =} \StringTok{"blue"}\NormalTok{) }\OperatorTok{+}
\StringTok{  }\KeywordTok{geom_point}\NormalTok{(}\DataTypeTok{data =}\NormalTok{ data_test, }\KeywordTok{aes}\NormalTok{(CAD_Weight, Actual_Weight), }\DataTypeTok{col =} \StringTok{"red"}\NormalTok{) }\OperatorTok{+}
\StringTok{  }\KeywordTok{labs}\NormalTok{(}\DataTypeTok{subtitle=}\StringTok{"Test Data"}\NormalTok{, }
       \DataTypeTok{y=}\StringTok{"Actual Weight"}\NormalTok{, }
       \DataTypeTok{x=}\StringTok{"CAD Weight"}\NormalTok{, }
       \DataTypeTok{title=}\StringTok{"Prediction Interval of Model 3 with Scatter Plot of Actual Weight vs CAD Weight"}\NormalTok{)}
\end{Highlighting}
\end{Shaded}

\includegraphics{CWAKnit_files/figure-latex/unnamed-chunk-5-1.pdf}

\begin{center}\rule{0.5\linewidth}{\linethickness}\end{center}

\section{Task 5: Estimating Model 5}\label{task-5-estimating-model-5}

\begin{Shaded}
\begin{Highlighting}[]
\NormalTok{formulas_}\DecValTok{5}\NormalTok{ <-}\StringTok{ }\KeywordTok{list}\NormalTok{(}\DataTypeTok{E =} \OperatorTok{~}\StringTok{ }\DecValTok{1} \OperatorTok{+}\StringTok{ }\NormalTok{CAD_Weight, }\DataTypeTok{V =} \OperatorTok{~}\StringTok{ }\DecValTok{1}\OperatorTok{+}\StringTok{ }\NormalTok{CAD_Weight)}

\NormalTok{estimates_}\DecValTok{5}\NormalTok{ <-}\StringTok{ }\KeywordTok{model_estimate}\NormalTok{(formulas_}\DecValTok{5}\NormalTok{, data_obs, }\StringTok{"Actual_Weight"}\NormalTok{)}

\NormalTok{estimates_}\DecValTok{5}
\end{Highlighting}
\end{Shaded}

\begin{verbatim}
## $theta
## [1] -0.16389778  1.08222367 -1.80995547  0.05356663
## 
## $formulas
## $formulas$E
## ~1 + CAD_Weight
## 
## $formulas$V
## ~1 + CAD_Weight
## 
## 
## $Sigma_theta
##               [,1]          [,2]          [,3]          [,4]
## [1,]  0.0237301631 -8.276727e-04 -1.111045e-03  3.112610e-05
## [2,] -0.0008276727  4.699560e-05  6.293605e-05 -1.763996e-06
## [3,] -0.0011110445  6.293605e-05  1.487803e-01 -3.347233e-03
## [4,]  0.0000311261 -1.763996e-06 -3.347233e-03  9.385984e-05
\end{verbatim}

\begin{center}\rule{0.5\linewidth}{\linethickness}\end{center}

\section{Task 6: Model 3 \& Model 5 Prediction Intervals
Plot}\label{task-6-model-3-model-5-prediction-intervals-plot}

\begin{Shaded}
\begin{Highlighting}[]
\NormalTok{model5 <-}\StringTok{ }\KeywordTok{model_predict}\NormalTok{(estimates_}\DecValTok{5}\NormalTok{[[}\StringTok{"theta"}\NormalTok{]], estimates_}\DecValTok{5}\NormalTok{[[}\StringTok{"formulas"}\NormalTok{]], estimates_}\DecValTok{5}\NormalTok{[[}\StringTok{"Sigma_theta"}\NormalTok{]],data_test,}\DataTypeTok{type=}\StringTok{"observation"}\NormalTok{)}


\NormalTok{pred_plot_}\DecValTok{6}\NormalTok{ <-}\StringTok{ }\KeywordTok{cbind}\NormalTok{(data_test, model5)}

\KeywordTok{ggplot}\NormalTok{() }\OperatorTok{+}
\StringTok{  }\KeywordTok{geom_ribbon}\NormalTok{(}\DataTypeTok{data =}\NormalTok{ pred_plot_}\DecValTok{4}\NormalTok{,}
              \KeywordTok{aes}\NormalTok{(CAD_Weight, }\DataTypeTok{ymin =}\NormalTok{ lwr, }\DataTypeTok{ymax =}\NormalTok{ upr,}\DataTypeTok{col=}\StringTok{"Model 3"}\NormalTok{,}\DataTypeTok{fill =} \StringTok{"Model 3"}\NormalTok{),}
              \DataTypeTok{alpha =} \FloatTok{0.25}\NormalTok{) }\OperatorTok{+}
\StringTok{  }\KeywordTok{geom_ribbon}\NormalTok{(}\DataTypeTok{data =}\NormalTok{ pred_plot_}\DecValTok{6}\NormalTok{,}
              \KeywordTok{aes}\NormalTok{(CAD_Weight, }\DataTypeTok{ymin =}\NormalTok{ lwr, }\DataTypeTok{ymax =}\NormalTok{ upr, }\DataTypeTok{col=}\StringTok{"Model 5"}\NormalTok{, }\DataTypeTok{fill =} \StringTok{"Model 5"}\NormalTok{),}
              \DataTypeTok{alpha =} \FloatTok{0.25}\NormalTok{) }\OperatorTok{+}
\StringTok{  }\KeywordTok{geom_line}\NormalTok{(}\DataTypeTok{data =}\NormalTok{ pred_plot_}\DecValTok{4}\NormalTok{, }\KeywordTok{aes}\NormalTok{(CAD_Weight, mu), }\DataTypeTok{col =} \StringTok{"blue"}\NormalTok{) }\OperatorTok{+}
\StringTok{  }\KeywordTok{geom_line}\NormalTok{(}\DataTypeTok{data =}\NormalTok{ pred_plot_}\DecValTok{6}\NormalTok{, }\KeywordTok{aes}\NormalTok{(CAD_Weight, mu), }\DataTypeTok{col =} \StringTok{"red"}\NormalTok{) }\OperatorTok{+}
\StringTok{  }\KeywordTok{geom_point}\NormalTok{(}\DataTypeTok{data =}\NormalTok{ data_test, }\KeywordTok{aes}\NormalTok{(CAD_Weight, Actual_Weight), }\DataTypeTok{col =} \StringTok{"black"}\NormalTok{)}\OperatorTok{+}
\StringTok{  }\KeywordTok{scale_colour_manual}\NormalTok{(}\StringTok{""}\NormalTok{,}\DataTypeTok{values =} \KeywordTok{c}\NormalTok{(}\StringTok{"Model 3"}\NormalTok{ =}\StringTok{ "blue"}\NormalTok{,}\StringTok{"Model 5"}\NormalTok{ =}\StringTok{ "red"}\NormalTok{))}\OperatorTok{+}
\StringTok{  }\KeywordTok{scale_fill_manual}\NormalTok{(}\StringTok{""}\NormalTok{,}\DataTypeTok{values =} \KeywordTok{c}\NormalTok{(}\StringTok{"Model 3"}\NormalTok{ =}\StringTok{ "blue"}\NormalTok{, }\StringTok{"Model 5"}\NormalTok{ =}\StringTok{ "red"}\NormalTok{))}\OperatorTok{+}
\StringTok{  }\KeywordTok{labs}\NormalTok{(}\DataTypeTok{subtitle=}\StringTok{"Test Data"}\NormalTok{, }
       \DataTypeTok{y=}\StringTok{"Actual Weight"}\NormalTok{, }
       \DataTypeTok{x=}\StringTok{"CAD Weight"}\NormalTok{, }
       \DataTypeTok{title=}\StringTok{"Prediction Interval of Model 3 and Model 5"}\NormalTok{)}
\end{Highlighting}
\end{Shaded}

\includegraphics{CWAKnit_files/figure-latex/unnamed-chunk-7-1.pdf}

\begin{center}\rule{0.5\linewidth}{\linethickness}\end{center}

\section{Task 7: Model 3 and Model 5 Score
Comparison}\label{task-7-model-3-and-model-5-score-comparison}

\begin{Shaded}
\begin{Highlighting}[]
\NormalTok{model_scores <-}\StringTok{ }\ControlFlowTok{function}\NormalTok{ (modelQ, data, response, alpha)\{}
  
\KeywordTok{return}\NormalTok{(}\KeywordTok{data.frame}\NormalTok{(}
  \DataTypeTok{SES=}\KeywordTok{c}\NormalTok{((}\KeywordTok{score_se}\NormalTok{(modelQ, data[[response]]))),  }\CommentTok{#Squared Error}
  \DataTypeTok{DSS=}\KeywordTok{c}\NormalTok{((}\KeywordTok{score_ds}\NormalTok{(modelQ, data[[response]]))),  }\CommentTok{#Dawid-Sebastiani}
  \DataTypeTok{IS=}\KeywordTok{c}\NormalTok{((}\KeywordTok{score_interval}\NormalTok{(modelQ, data[[response]], }\DataTypeTok{alpha =}\NormalTok{ alpha))) }\CommentTok{#Interval Score}
\NormalTok{))  }
\NormalTok{\}}

\NormalTok{Score_}\DecValTok{3}\NormalTok{ =}\StringTok{ }\KeywordTok{model_scores}\NormalTok{(model3, data_test, }\StringTok{'Actual_Weight'}\NormalTok{, }\FloatTok{0.1}\NormalTok{)}
\NormalTok{Score_}\DecValTok{5}\NormalTok{ =}\StringTok{ }\KeywordTok{model_scores}\NormalTok{(model5, data_test, }\StringTok{'Actual_Weight'}\NormalTok{, }\FloatTok{0.1}\NormalTok{)}

\NormalTok{S5_S3 =}\StringTok{ }\NormalTok{Score_}\DecValTok{5} \OperatorTok{-}\StringTok{ }\NormalTok{Score_}\DecValTok{3}

\KeywordTok{plot}\NormalTok{(}\KeywordTok{ecdf}\NormalTok{(S5_S3[[}\StringTok{"SES"}\NormalTok{]]),}
      \DataTypeTok{xlim=}\KeywordTok{c}\NormalTok{(}\OperatorTok{-}\DecValTok{4}\NormalTok{,}\DecValTok{4}\NormalTok{),}
      \DataTypeTok{xlab=}\StringTok{"Score Difference (Score 5 - Score 3)"}\NormalTok{,}
      \DataTypeTok{ylab=}\StringTok{"% Test Data Set"}\NormalTok{,}
      \DataTypeTok{main=}\StringTok{"ECDF of Difference between Scores of Model 5 and Model 3"}\NormalTok{,}
      \DataTypeTok{col=}\StringTok{"red"}\NormalTok{,}\DataTypeTok{cex=}\DecValTok{0}\NormalTok{)}
\KeywordTok{lines}\NormalTok{(}\KeywordTok{ecdf}\NormalTok{(S5_S3[[}\StringTok{"DSS"}\NormalTok{]]),}
     \DataTypeTok{col=}\StringTok{"blue"}\NormalTok{,}\DataTypeTok{cex=}\DecValTok{0}\NormalTok{)}
\KeywordTok{lines}\NormalTok{(}\KeywordTok{ecdf}\NormalTok{(S5_S3[[}\StringTok{"IS"}\NormalTok{]]),}
     \DataTypeTok{col=}\StringTok{"black"}\NormalTok{,}\DataTypeTok{cex=}\DecValTok{0}\NormalTok{)}
\KeywordTok{abline}\NormalTok{(}\DataTypeTok{v=}\DecValTok{0}\NormalTok{, }\DataTypeTok{col=}\StringTok{"magenta"}\NormalTok{)}

\KeywordTok{legend}\NormalTok{(}\StringTok{'bottomright'}\NormalTok{, }
       \DataTypeTok{legend=}\KeywordTok{c}\NormalTok{(}\StringTok{"Squared Error"}\NormalTok{,}\StringTok{"Dawid-Sebastiani"}\NormalTok{,}\StringTok{"Interval Score"}\NormalTok{),}
       \DataTypeTok{col=}\KeywordTok{c}\NormalTok{(}\StringTok{"red"}\NormalTok{,}\StringTok{"blue"}\NormalTok{,}\StringTok{"black"}\NormalTok{),}
       \DataTypeTok{pch=}\DecValTok{15}\NormalTok{)}
\end{Highlighting}
\end{Shaded}

\includegraphics{CWAKnit_files/figure-latex/unnamed-chunk-8-1.pdf}

\subsubsection{Comparing Model 3 and Model
5}\label{comparing-model-3-and-model-5}

The interval score for Model 5 is better than Model 3 on appoximately
80\% of the test data set. The Dawid-Sebastiani score for Model 5 is
better than Model 3 on approximately 60\% of the test data set. The
Squared-Error score for Model 5 is better than Model 3 on approximately
35\% of the test data set.

Only the interval score and Dawid-Sebastiani score agree on Model 5
being better than Model 3. The Squared-Error score suggest that Model 3
is marginally better than Model 5.

\begin{center}\rule{0.5\linewidth}{\linethickness}\end{center}

\section{Task 8}\label{task-8}

\begin{Shaded}
\begin{Highlighting}[]
\NormalTok{formulas_}\DecValTok{8}\NormalTok{ <-}\StringTok{ }\KeywordTok{list}\NormalTok{(}\DataTypeTok{E =} \OperatorTok{~}\StringTok{ }\DecValTok{1} \OperatorTok{+}\StringTok{ }\NormalTok{CAD_Weight}\OperatorTok{:}\NormalTok{Material, }\DataTypeTok{V =} \OperatorTok{~}\StringTok{ }\DecValTok{1}\OperatorTok{+}\StringTok{ }\NormalTok{CAD_Weight)}

\NormalTok{estimates_}\DecValTok{8}\NormalTok{ <-}\StringTok{ }\KeywordTok{model_estimate}\NormalTok{(formulas_}\DecValTok{8}\NormalTok{, data_obs, }\StringTok{"Actual_Weight"}\NormalTok{)}

\NormalTok{estimates_}\DecValTok{8}
\end{Highlighting}
\end{Shaded}

\begin{verbatim}
## $theta
## [1] -0.07467418  1.06083919  1.08529823  1.10359160  1.06565852  1.09929646
## [7]  1.07417646 -1.89254757  0.05068071
## 
## $formulas
## $formulas$E
## ~1 + CAD_Weight:Material
## 
## $formulas$V
## ~1 + CAD_Weight
## 
## 
## $Sigma_theta
##                [,1]          [,2]          [,3]          [,4]
##  [1,]  0.0220436724 -8.933195e-04 -7.814548e-04 -5.665713e-04
##  [2,] -0.0008933195  9.998818e-05  3.139510e-05  2.235492e-05
##  [3,] -0.0007814548  3.139510e-05  8.027081e-05  2.023356e-05
##  [4,] -0.0005665713  2.235492e-05  2.023356e-05  1.129322e-04
##  [5,] -0.0009598999  3.770394e-05  3.432199e-05  2.532102e-05
##  [6,] -0.0006035296  2.811462e-05  2.049873e-05  1.352638e-05
##  [7,] -0.0006636840  2.687504e-05  2.353288e-05  1.706942e-05
##  [8,]  0.0030452444 -4.922692e-04 -1.783771e-05  1.219876e-04
##  [9,] -0.0000853920  1.380483e-05  4.908143e-07 -3.422572e-06
##                [,5]          [,6]          [,7]          [,8]
##  [1,] -9.598999e-04 -6.035296e-04 -6.636840e-04  3.045244e-03
##  [2,]  3.770394e-05  2.811462e-05  2.687504e-05 -4.922692e-04
##  [3,]  3.432199e-05  2.049873e-05  2.353288e-05 -1.783771e-05
##  [4,]  2.532102e-05  1.352638e-05  1.706942e-05  1.219876e-04
##  [5,]  3.070196e-04  2.235800e-05  2.892262e-05  2.631421e-04
##  [6,]  2.235800e-05  3.959546e-04  1.810284e-05 -1.293340e-03
##  [7,]  2.892262e-05  1.810284e-05  8.642906e-05 -8.482079e-05
##  [8,]  2.631421e-04 -1.293340e-03 -8.482079e-05  1.519116e-01
##  [9,] -7.379281e-06  3.626801e-05  2.378478e-06 -3.435042e-03
##                [,9]
##  [1,] -8.539200e-05
##  [2,]  1.380483e-05
##  [3,]  4.908143e-07
##  [4,] -3.422572e-06
##  [5,] -7.379281e-06
##  [6,]  3.626801e-05
##  [7,]  2.378478e-06
##  [8,] -3.435042e-03
##  [9,]  9.631711e-05
\end{verbatim}

\begin{Shaded}
\begin{Highlighting}[]
\NormalTok{model8 <-}\StringTok{ }\KeywordTok{model_predict}\NormalTok{(estimates_}\DecValTok{8}\NormalTok{[[}\StringTok{"theta"}\NormalTok{]], estimates_}\DecValTok{8}\NormalTok{[[}\StringTok{"formulas"}\NormalTok{]], estimates_}\DecValTok{8}\NormalTok{[[}\StringTok{"Sigma_theta"}\NormalTok{]],data_test,}\DataTypeTok{type=}\StringTok{"observation"}\NormalTok{)}


\NormalTok{Score_}\DecValTok{8}\NormalTok{ =}\StringTok{ }\KeywordTok{model_scores}\NormalTok{(model8, data_test, }\StringTok{'Actual_Weight'}\NormalTok{, }\FloatTok{0.1}\NormalTok{)}

\NormalTok{S8_S3 =}\StringTok{ }\NormalTok{Score_}\DecValTok{8} \OperatorTok{-}\StringTok{ }\NormalTok{Score_}\DecValTok{3}
\NormalTok{S8_S5 =}\StringTok{ }\NormalTok{Score_}\DecValTok{8} \OperatorTok{-}\StringTok{ }\NormalTok{Score_}\DecValTok{5}

\KeywordTok{plot}\NormalTok{(}\KeywordTok{ecdf}\NormalTok{(S8_S3[[}\StringTok{"SES"}\NormalTok{]]),}
      \DataTypeTok{xlim=}\KeywordTok{c}\NormalTok{(}\OperatorTok{-}\DecValTok{2}\NormalTok{,}\DecValTok{2}\NormalTok{),}
      \DataTypeTok{xlab=}\StringTok{"Score Difference (Score 8 - Score 3)"}\NormalTok{,}
      \DataTypeTok{ylab=}\StringTok{"% Test Data Set"}\NormalTok{,}
      \DataTypeTok{main=}\StringTok{"ECDF of Difference between Scores of Model 8 and Model 3"}\NormalTok{,}
      \DataTypeTok{col=}\StringTok{"red"}\NormalTok{,}\DataTypeTok{cex=}\DecValTok{0}\NormalTok{)}
\KeywordTok{lines}\NormalTok{(}\KeywordTok{ecdf}\NormalTok{(S8_S3[[}\StringTok{"DSS"}\NormalTok{]]),}
     \DataTypeTok{col=}\StringTok{"blue"}\NormalTok{,}\DataTypeTok{cex=}\DecValTok{0}\NormalTok{)}
\KeywordTok{lines}\NormalTok{(}\KeywordTok{ecdf}\NormalTok{(S8_S3[[}\StringTok{"IS"}\NormalTok{]]),}
     \DataTypeTok{col=}\StringTok{"black"}\NormalTok{,}\DataTypeTok{cex=}\DecValTok{0}\NormalTok{)}
\KeywordTok{abline}\NormalTok{(}\DataTypeTok{v=}\DecValTok{0}\NormalTok{, }\DataTypeTok{col=}\StringTok{"magenta"}\NormalTok{)}
\KeywordTok{legend}\NormalTok{(}\StringTok{'bottomright'}\NormalTok{, }
       \DataTypeTok{legend=}\KeywordTok{c}\NormalTok{(}\StringTok{"Squared Error"}\NormalTok{,}\StringTok{"Dawid-Sebastiani"}\NormalTok{,}\StringTok{"Interval Score"}\NormalTok{),}
       \DataTypeTok{col=}\KeywordTok{c}\NormalTok{(}\StringTok{"red"}\NormalTok{,}\StringTok{"blue"}\NormalTok{,}\StringTok{"black"}\NormalTok{),}
       \DataTypeTok{pch=}\DecValTok{15}\NormalTok{)}
\end{Highlighting}
\end{Shaded}

\includegraphics{CWAKnit_files/figure-latex/unnamed-chunk-10-1.pdf}

\subsubsection{Comparing Model 8 and Model
3}\label{comparing-model-8-and-model-3}

The interval score for Model 8 is better than Model 3 on appoximately
\textbf{75\%} of the test data set. The Dawid-Sebastiani score for Model
8 is better than Model 3 on approximately \textbf{70\%} of the test data
set. The Squared-Error score for Model 8 is better than Model 3 on
approximately \textbf{40\%} of the test data set.

Only the interval score and Dawid-Sebastiani score agree on Model 8
being better than Model 3. The Squared-Error score suggest that Model 3
is marginally better than Model 8.

\begin{Shaded}
\begin{Highlighting}[]
\KeywordTok{plot}\NormalTok{(}\KeywordTok{ecdf}\NormalTok{(S8_S5[[}\StringTok{"SES"}\NormalTok{]]),}
      \DataTypeTok{xlim=}\KeywordTok{c}\NormalTok{(}\OperatorTok{-}\DecValTok{2}\NormalTok{,}\DecValTok{2}\NormalTok{),}
      \DataTypeTok{xlab=}\StringTok{"Score Difference (Score 8 - Score 5)"}\NormalTok{,}
      \DataTypeTok{ylab=}\StringTok{"% Test Data Set"}\NormalTok{,}
      \DataTypeTok{main=}\StringTok{"ECDF of Difference between Scores of Model 8 and Model 5"}\NormalTok{,}
      \DataTypeTok{col=}\StringTok{"red"}\NormalTok{,}\DataTypeTok{cex=}\DecValTok{0}\NormalTok{)}
\KeywordTok{lines}\NormalTok{(}\KeywordTok{ecdf}\NormalTok{(S8_S5[[}\StringTok{"DSS"}\NormalTok{]]),}
     \DataTypeTok{col=}\StringTok{"blue"}\NormalTok{,}\DataTypeTok{cex=}\DecValTok{0}\NormalTok{)}
\KeywordTok{lines}\NormalTok{(}\KeywordTok{ecdf}\NormalTok{(S8_S5[[}\StringTok{"IS"}\NormalTok{]]),}
     \DataTypeTok{col=}\StringTok{"black"}\NormalTok{,}\DataTypeTok{cex=}\DecValTok{0}\NormalTok{)}
\KeywordTok{abline}\NormalTok{(}\DataTypeTok{v=}\DecValTok{0}\NormalTok{, }\DataTypeTok{col=}\StringTok{"magenta"}\NormalTok{)}
\KeywordTok{legend}\NormalTok{(}\StringTok{'bottomright'}\NormalTok{, }
       \DataTypeTok{legend=}\KeywordTok{c}\NormalTok{(}\StringTok{"Squared Error"}\NormalTok{,}\StringTok{"Dawid-Sebastiani"}\NormalTok{,}\StringTok{"Interval Score"}\NormalTok{),}
       \DataTypeTok{col=}\KeywordTok{c}\NormalTok{(}\StringTok{"red"}\NormalTok{,}\StringTok{"blue"}\NormalTok{,}\StringTok{"black"}\NormalTok{),}
       \DataTypeTok{pch=}\DecValTok{15}\NormalTok{)}
\end{Highlighting}
\end{Shaded}

\includegraphics{CWAKnit_files/figure-latex/unnamed-chunk-11-1.pdf}

\subsubsection{Comparing Model 8 and Model
5}\label{comparing-model-8-and-model-5}

The interval score for Model 8 is better than Model 5 on appoximately
\textbf{80\%} of the test data set. The Dawid-Sebastiani score for Model
8 is better than Model 5 on approximately \textbf{50\%} of the test data
set. The Squared-Error score for Model 8 is better than Model 5 on
approximately \textbf{35\%} of the test data set.

Only the interval score agrees on Model 8 being better than Model 5. The
Dawid-Sebastiani score dow not give a conclusive indication on which
model is better. The Squared-Error score suggest that Model 5 is better
than Model 8.

\begin{center}\rule{0.5\linewidth}{\linethickness}\end{center}

\section{Task 9}\label{task-9}

\begin{Shaded}
\begin{Highlighting}[]
\CommentTok{#Estimating Probability Distributions}
\NormalTok{Prob_}\DecValTok{3}\NormalTok{ =}\StringTok{ }\KeywordTok{pnorm}\NormalTok{(}\DataTypeTok{q=}\FloatTok{1.1}\OperatorTok{*}\NormalTok{data_test}\OperatorTok{$}\NormalTok{CAD_Weight,}\DataTypeTok{mean =}\NormalTok{ model3}\OperatorTok{$}\NormalTok{mu,}\DataTypeTok{sd =}\NormalTok{ model3}\OperatorTok{$}\NormalTok{sigma,}\DataTypeTok{lower.tail =} \OtherTok{FALSE}\NormalTok{)}
\NormalTok{Prob_}\DecValTok{5}\NormalTok{ =}\StringTok{ }\KeywordTok{pnorm}\NormalTok{(}\DataTypeTok{q=}\FloatTok{1.1}\OperatorTok{*}\NormalTok{data_test}\OperatorTok{$}\NormalTok{CAD_Weight,}\DataTypeTok{mean =}\NormalTok{ model5}\OperatorTok{$}\NormalTok{mu,}\DataTypeTok{sd =}\NormalTok{ model5}\OperatorTok{$}\NormalTok{sigma,}\DataTypeTok{lower.tail =} \OtherTok{FALSE}\NormalTok{)}
\NormalTok{Prob_}\DecValTok{8}\NormalTok{ =}\StringTok{ }\KeywordTok{pnorm}\NormalTok{(}\DataTypeTok{q=}\FloatTok{1.1}\OperatorTok{*}\NormalTok{data_test}\OperatorTok{$}\NormalTok{CAD_Weight,}\DataTypeTok{mean =}\NormalTok{ model8}\OperatorTok{$}\NormalTok{mu,}\DataTypeTok{sd =}\NormalTok{ model8}\OperatorTok{$}\NormalTok{sigma,}\DataTypeTok{lower.tail =} \OtherTok{FALSE}\NormalTok{)}

\NormalTok{Prob_CAD=}\KeywordTok{cbind}\NormalTok{(Prob_}\DecValTok{3}\NormalTok{,Prob_}\DecValTok{5}\NormalTok{,Prob_}\DecValTok{8}\NormalTok{,data_test)}

\KeywordTok{ggplot}\NormalTok{() }\OperatorTok{+}
\StringTok{  }\KeywordTok{geom_point}\NormalTok{(}\DataTypeTok{data=}\NormalTok{Prob_CAD,}\KeywordTok{aes}\NormalTok{(CAD_Weight,Prob_}\DecValTok{3}\NormalTok{, }\DataTypeTok{col =} \StringTok{"Model 3"}\NormalTok{ )) }\OperatorTok{+}
\StringTok{  }\KeywordTok{geom_point}\NormalTok{(}\DataTypeTok{data=}\NormalTok{Prob_CAD,}\KeywordTok{aes}\NormalTok{(CAD_Weight,Prob_}\DecValTok{5}\NormalTok{, }\DataTypeTok{col =} \StringTok{"Model 5"}\NormalTok{)) }\OperatorTok{+}
\StringTok{  }\KeywordTok{geom_line}\NormalTok{(}\DataTypeTok{data=}\NormalTok{Prob_CAD,}\KeywordTok{aes}\NormalTok{(CAD_Weight,Prob_}\DecValTok{3}\NormalTok{, }\DataTypeTok{col =} \StringTok{"Model 3"}\NormalTok{ )) }\OperatorTok{+}
\StringTok{  }\KeywordTok{geom_line}\NormalTok{(}\DataTypeTok{data=}\NormalTok{Prob_CAD,}\KeywordTok{aes}\NormalTok{(CAD_Weight,Prob_}\DecValTok{5}\NormalTok{, }\DataTypeTok{col =} \StringTok{"Model 5"}\NormalTok{)) }\OperatorTok{+}
\StringTok{  }\KeywordTok{geom_point}\NormalTok{(}\DataTypeTok{data=}\NormalTok{Prob_CAD,}\KeywordTok{aes}\NormalTok{(CAD_Weight,Prob_}\DecValTok{8}\NormalTok{, }\DataTypeTok{col =} \StringTok{"Model 8"}\NormalTok{)) }\OperatorTok{+}
\StringTok{  }\KeywordTok{labs}\NormalTok{(}\DataTypeTok{subtitle=}\StringTok{"Event: More than 10% extra weight is needed compared with CAD_Weight"}\NormalTok{, }
       \DataTypeTok{y=}\StringTok{"Probability"}\NormalTok{, }
       \DataTypeTok{x=}\StringTok{"CAD_Weight"}\NormalTok{, }
       \DataTypeTok{title=}\StringTok{"Probabilities for the Event for each Model"}\NormalTok{, }
       \DataTypeTok{caption =} \StringTok{"Note: Probabilities for Model 8 shown as Scatter plot only"}\NormalTok{)}
\end{Highlighting}
\end{Shaded}

\includegraphics{CWAKnit_files/figure-latex/unnamed-chunk-12-1.pdf}

\begin{Shaded}
\begin{Highlighting}[]
\CommentTok{#Brier Score Function}
\NormalTok{score_brier <-}\StringTok{ }\ControlFlowTok{function}\NormalTok{ (z, probF)\{}
\NormalTok{  (z }\OperatorTok{-}\StringTok{ }\NormalTok{probF)}\OperatorTok{^}\DecValTok{2}
\NormalTok{\}}

\CommentTok{# if Actual_Weight < 1.1*CAD_Weight then z=1 otherwise z=0}
\NormalTok{indicator<-}\StringTok{ }\KeywordTok{ifelse}\NormalTok{(data_test}\OperatorTok{$}\NormalTok{Actual_Weight}\OperatorTok{>}\NormalTok{data_test}\OperatorTok{$}\NormalTok{CAD_Weight}\OperatorTok{*}\FloatTok{1.1}\NormalTok{,}\DecValTok{1}\NormalTok{,}\DecValTok{0}\NormalTok{)}

\NormalTok{BS_}\DecValTok{3}\NormalTok{=}\KeywordTok{c}\NormalTok{((}\KeywordTok{score_brier}\NormalTok{(indicator,Prob_}\DecValTok{3}\NormalTok{)))}
\NormalTok{BS_}\DecValTok{5}\NormalTok{=}\KeywordTok{c}\NormalTok{((}\KeywordTok{score_brier}\NormalTok{(indicator,Prob_}\DecValTok{5}\NormalTok{)))}
\NormalTok{BS_}\DecValTok{8}\NormalTok{=}\KeywordTok{c}\NormalTok{((}\KeywordTok{score_brier}\NormalTok{(indicator,Prob_}\DecValTok{8}\NormalTok{)))}

\NormalTok{df.brier=}\KeywordTok{data.frame}\NormalTok{(}\StringTok{"Brier Score for Model 3"}\NormalTok{=BS_}\DecValTok{3}\NormalTok{,}\StringTok{"Brier Score for Model 5"}\NormalTok{=BS_}\DecValTok{5}\NormalTok{,}\StringTok{"Brier Score for Model 8"}\NormalTok{=BS_}\DecValTok{8}\NormalTok{)}

\NormalTok{B5_B3 =}\StringTok{ }\NormalTok{BS_}\DecValTok{5}\OperatorTok{-}\NormalTok{BS_}\DecValTok{3}
\NormalTok{B8_B3 =}\StringTok{ }\NormalTok{BS_}\DecValTok{8}\OperatorTok{-}\NormalTok{BS_}\DecValTok{3}
\NormalTok{B8_B5 =}\StringTok{ }\NormalTok{BS_}\DecValTok{8}\OperatorTok{-}\NormalTok{BS_}\DecValTok{5}

\KeywordTok{plot}\NormalTok{(}\KeywordTok{ecdf}\NormalTok{(B8_B3),}
     \DataTypeTok{xlim=}\KeywordTok{c}\NormalTok{(}\OperatorTok{-}\FloatTok{0.2}\NormalTok{,}\FloatTok{0.2}\NormalTok{),}
      \DataTypeTok{xlab=}\StringTok{"Score Difference (Score 8 - Score 3)"}\NormalTok{,}
      \DataTypeTok{ylab=}\StringTok{"% Test Data Set"}\NormalTok{,}
      \DataTypeTok{main=}\StringTok{"ECDF of Brier Score Difference between Model 8, 5 & 3"}\NormalTok{,}
      \DataTypeTok{col=}\StringTok{"red"}\NormalTok{,}\DataTypeTok{cex=}\DecValTok{0}\NormalTok{)}
\KeywordTok{lines}\NormalTok{(}\KeywordTok{ecdf}\NormalTok{(B5_B3),}
     \DataTypeTok{col=}\StringTok{"blue"}\NormalTok{,}\DataTypeTok{cex=}\DecValTok{0}\NormalTok{)}
\KeywordTok{lines}\NormalTok{(}\KeywordTok{ecdf}\NormalTok{(B8_B5),}
     \DataTypeTok{col=}\StringTok{"black"}\NormalTok{,}\DataTypeTok{cex=}\DecValTok{0}\NormalTok{)}
\KeywordTok{abline}\NormalTok{(}\DataTypeTok{v=}\DecValTok{0}\NormalTok{, }\DataTypeTok{col=}\StringTok{"magenta"}\NormalTok{)}
\KeywordTok{legend}\NormalTok{(}\StringTok{'bottomright'}\NormalTok{, }
       \DataTypeTok{legend=}\KeywordTok{c}\NormalTok{(}\StringTok{"Model 8 - Model 3"}\NormalTok{,}\StringTok{"Model 5 - Model 3"}\NormalTok{,}\StringTok{"Model 8 - Model 5"}\NormalTok{),}
       \DataTypeTok{col=}\KeywordTok{c}\NormalTok{(}\StringTok{"red"}\NormalTok{,}\StringTok{"blue"}\NormalTok{,}\StringTok{"black"}\NormalTok{),}
       \DataTypeTok{pch=}\DecValTok{15}\NormalTok{)}
\end{Highlighting}
\end{Shaded}

\includegraphics{CWAKnit_files/figure-latex/unnamed-chunk-12-2.pdf}

\subsubsection{Comparing Model 8, Model 5 \& Model
3}\label{comparing-model-8-model-5-model-3}

The Brier score for Model 8 is better than Model 5 on appoximately
\textbf{50\%} of the test data set. The Brier score for Model 8 is
better than Model 3 on approximately \textbf{45\%} of the test data set.
The Brier score for Model 5 is better than Model 3 on approximately
\textbf{40\%} of the test data set.

The Brier score does not give a conclusive indication on which model is
better.

\begin{center}\rule{0.5\linewidth}{\linethickness}\end{center}

\section{Task 10}\label{task-10}

\begin{Shaded}
\begin{Highlighting}[]
\CommentTok{#set.seed(pi) #for reproducability}

\CommentTok{# Generating observed data}
\NormalTok{c_}\DecValTok{5}\NormalTok{=}\KeywordTok{rcauchy}\NormalTok{(}\DecValTok{5}\NormalTok{,}\DataTypeTok{location=}\DecValTok{2}\NormalTok{,}\DataTypeTok{scale=}\DecValTok{5}\NormalTok{)}
\NormalTok{c_}\DecValTok{10}\NormalTok{=}\KeywordTok{rcauchy}\NormalTok{(}\DecValTok{10}\NormalTok{,}\DataTypeTok{location=}\DecValTok{2}\NormalTok{,}\DataTypeTok{scale=}\DecValTok{5}\NormalTok{)}
\NormalTok{c_}\DecValTok{20}\NormalTok{=}\KeywordTok{rcauchy}\NormalTok{(}\DecValTok{20}\NormalTok{,}\DataTypeTok{location=}\DecValTok{2}\NormalTok{,}\DataTypeTok{scale=}\DecValTok{5}\NormalTok{)}
\NormalTok{c_}\DecValTok{40}\NormalTok{=}\KeywordTok{rcauchy}\NormalTok{(}\DecValTok{40}\NormalTok{,}\DataTypeTok{location=}\DecValTok{2}\NormalTok{,}\DataTypeTok{scale=}\DecValTok{5}\NormalTok{)}

\CommentTok{# Generating test data}
\NormalTok{c_test=}\KeywordTok{rcauchy}\NormalTok{(}\DecValTok{100000}\NormalTok{,}\DataTypeTok{location=}\DecValTok{2}\NormalTok{,}\DataTypeTok{scale=}\DecValTok{5}\NormalTok{)}
\end{Highlighting}
\end{Shaded}

\begin{Shaded}
\begin{Highlighting}[]
\NormalTok{neg_lik_cauchy <-}\StringTok{ }\ControlFlowTok{function}\NormalTok{(theta, y) \{}
  \OperatorTok{-}\KeywordTok{sum}\NormalTok{(}\KeywordTok{dcauchy}\NormalTok{(}
\NormalTok{    y, }
    \DataTypeTok{location =}\NormalTok{ theta[}\DecValTok{1}\NormalTok{],}
    \DataTypeTok{scale =} \KeywordTok{exp}\NormalTok{(theta[}\DecValTok{2}\NormalTok{])}\OperatorTok{^}\FloatTok{0.5}\NormalTok{, }\CommentTok{#transforming scale}
    \DataTypeTok{log =} \OtherTok{TRUE}\NormalTok{))}
\NormalTok{\}}


\NormalTok{opt_cauchy <-}\StringTok{ }\ControlFlowTok{function}\NormalTok{ (C_N)\{}
\NormalTok{  opt_c <-}\StringTok{ }\KeywordTok{optim}\NormalTok{(}\KeywordTok{c}\NormalTok{(}\DecValTok{0}\NormalTok{,}\DecValTok{0}\NormalTok{),}
              \DataTypeTok{fn =}\NormalTok{ neg_lik_cauchy,}
              \DataTypeTok{y =}\NormalTok{ C_N,}
              \DataTypeTok{method =} \StringTok{"BFGS"}\NormalTok{,}
              \DataTypeTok{control =} \KeywordTok{list}\NormalTok{(}\DataTypeTok{maxit =} \DecValTok{5000}\NormalTok{), }\CommentTok{# Anouncement on Learn, Ensures Convergence}
              \DataTypeTok{hessian =} \OtherTok{TRUE}\NormalTok{)}
  
\NormalTok{  location_e <-}\StringTok{ }\NormalTok{opt_c}\OperatorTok{$}\NormalTok{par[}\DecValTok{1}\NormalTok{]}
\NormalTok{  scale_e <-}\StringTok{ }\NormalTok{opt_c}\OperatorTok{$}\NormalTok{par[}\DecValTok{2}\NormalTok{]}
  
  \KeywordTok{return}\NormalTok{ (}\KeywordTok{list}\NormalTok{(}\DataTypeTok{location_e=}\NormalTok{location_e,}\DataTypeTok{scale_e=}\NormalTok{scale_e, }\DataTypeTok{convergence=}\NormalTok{opt_c}\OperatorTok{$}\NormalTok{convergence))}
\NormalTok{\}}

\NormalTok{brier_cauchy <-}\StringTok{ }\ControlFlowTok{function}\NormalTok{ (C_N)\{}
\NormalTok{  opt_c <-}\StringTok{ }\KeywordTok{opt_cauchy}\NormalTok{(C_N)}
  
  \CommentTok{#Estimated Probability Distribution}
\NormalTok{  prob_dist_e <-}\StringTok{ }\KeywordTok{pcauchy}\NormalTok{(}\DataTypeTok{q=}\DecValTok{0}\NormalTok{,}\DataTypeTok{location =}\NormalTok{ opt_c}\OperatorTok{$}\NormalTok{location_e, }\DataTypeTok{scale=}\NormalTok{opt_c}\OperatorTok{$}\NormalTok{scale_e,}\DataTypeTok{lower.tail =} \OtherTok{TRUE}\NormalTok{)}
  \CommentTok{#True Probability Distribution}
\NormalTok{  prob_dist_true =}\StringTok{ }\KeywordTok{pcauchy}\NormalTok{(}\DataTypeTok{q=}\DecValTok{0}\NormalTok{,}\DataTypeTok{location =} \DecValTok{2}\NormalTok{, }\DataTypeTok{scale=}\DecValTok{5}\NormalTok{,}\DataTypeTok{lower.tail =} \OtherTok{TRUE}\NormalTok{) }
  
  
\NormalTok{  indicator_e<-}\StringTok{ }\KeywordTok{ifelse}\NormalTok{(c_test}\OperatorTok{<}\DecValTok{0}\NormalTok{,}\DecValTok{1}\NormalTok{,}\DecValTok{0}\NormalTok{) }\CommentTok{#if y<0 then 1 else 0}
\NormalTok{  indicator_true<-}\StringTok{ }\KeywordTok{ifelse}\NormalTok{(C_N}\OperatorTok{<}\DecValTok{0}\NormalTok{,}\DecValTok{1}\NormalTok{,}\DecValTok{0}\NormalTok{)}
  
\NormalTok{  BS_10_e=}\KeywordTok{c}\NormalTok{((}\KeywordTok{score_brier}\NormalTok{(indicator_e,prob_dist_e))) }\CommentTok{#using  score_brier from Task 9}
\NormalTok{  BS_10_true=}\KeywordTok{c}\NormalTok{((}\KeywordTok{score_brier}\NormalTok{(indicator_true,prob_dist_true))) }
  
\NormalTok{  mean_BS_10_e=}\KeywordTok{mean}\NormalTok{(BS_10_e) }\CommentTok{#Estimated Mean Brier Score}
\NormalTok{  mean_BS_10_true=}\KeywordTok{mean}\NormalTok{(BS_10_true) }\CommentTok{#True Mean Brier Score}
  
\NormalTok{  mean_bs_d=mean_BS_10_e }\OperatorTok{-}\StringTok{ }\NormalTok{mean_BS_10_true }\CommentTok{#Difference between scores (Estimated - True)}

  \KeywordTok{return}\NormalTok{(}\KeywordTok{list}\NormalTok{(}\DataTypeTok{mean_bs_e=}\NormalTok{mean_BS_10_e, }\DataTypeTok{mean_bs_true=}\NormalTok{mean_BS_10_true, }\DataTypeTok{mean_d=}\NormalTok{mean_bs_d,}\DataTypeTok{opt_c=}\NormalTok{opt_c))}
\NormalTok{\}}


\NormalTok{b_}\DecValTok{5}\NormalTok{=}\KeywordTok{brier_cauchy}\NormalTok{(c_}\DecValTok{5}\NormalTok{)}
\NormalTok{b_}\DecValTok{10}\NormalTok{=}\KeywordTok{brier_cauchy}\NormalTok{(c_}\DecValTok{10}\NormalTok{)}
\NormalTok{b_}\DecValTok{20}\NormalTok{=}\KeywordTok{brier_cauchy}\NormalTok{(c_}\DecValTok{20}\NormalTok{)}
\NormalTok{b_}\DecValTok{40}\NormalTok{=}\KeywordTok{brier_cauchy}\NormalTok{(c_}\DecValTok{40}\NormalTok{)}

\NormalTok{mbs_n=}\KeywordTok{data.frame}\NormalTok{(}\KeywordTok{cbind}\NormalTok{(}\DataTypeTok{mean_d=}\KeywordTok{c}\NormalTok{(b_}\DecValTok{5}\OperatorTok{$}\NormalTok{mean_d,b_}\DecValTok{10}\OperatorTok{$}\NormalTok{mean_d,b_}\DecValTok{20}\OperatorTok{$}\NormalTok{mean_d,b_}\DecValTok{40}\OperatorTok{$}\NormalTok{mean_d),}\DataTypeTok{N=}\KeywordTok{c}\NormalTok{(}\DecValTok{5}\NormalTok{,}\DecValTok{10}\NormalTok{,}\DecValTok{20}\NormalTok{,}\DecValTok{40}\NormalTok{)))}

\KeywordTok{ggplot}\NormalTok{()}\OperatorTok{+}
\StringTok{  }\KeywordTok{geom_line}\NormalTok{(}\DataTypeTok{data=}\NormalTok{mbs_n,}\KeywordTok{aes}\NormalTok{(N,mean_d))}\OperatorTok{+}
\StringTok{  }\KeywordTok{geom_point}\NormalTok{(}\DataTypeTok{data=}\NormalTok{mbs_n,}\KeywordTok{aes}\NormalTok{(N,mean_d))}\OperatorTok{+}
\StringTok{  }\KeywordTok{geom_hline}\NormalTok{(}\DataTypeTok{yintercept =} \DecValTok{0}\NormalTok{, }\DataTypeTok{col=}\StringTok{"red"}\NormalTok{)}\OperatorTok{+}
\StringTok{  }\KeywordTok{labs}\NormalTok{(}\DataTypeTok{y=}\StringTok{"Mean Brier Score Difference "}\NormalTok{, }
       \DataTypeTok{x=}\StringTok{"Number of Observations"}\NormalTok{, }
       \DataTypeTok{title=}\StringTok{"Mean Brier Score Difference Stability"}\NormalTok{)}
\end{Highlighting}
\end{Shaded}

\includegraphics{CWAKnit_files/figure-latex/unnamed-chunk-14-1.pdf}

\subsubsection{Stabilisation of Mean Brier Score
Differences}\label{stabilisation-of-mean-brier-score-differences}

The mean Brier Score difference stabilise as N increases. The difference
approaches 0 as N increases.

\begin{Shaded}
\begin{Highlighting}[]
\NormalTok{param_n=}\KeywordTok{data.frame}\NormalTok{(}\KeywordTok{cbind}\NormalTok{(}\DataTypeTok{opt_par_l=}\KeywordTok{c}\NormalTok{(b_}\DecValTok{5}\OperatorTok{$}\NormalTok{opt_c}\OperatorTok{$}\NormalTok{location_e,b_}\DecValTok{10}\OperatorTok{$}\NormalTok{opt_c}\OperatorTok{$}\NormalTok{location_e,b_}\DecValTok{20}\OperatorTok{$}\NormalTok{opt_c}\OperatorTok{$}\NormalTok{location_e,b_}\DecValTok{40}\OperatorTok{$}\NormalTok{opt_c}\OperatorTok{$}\NormalTok{location_e),}\KeywordTok{cbind}\NormalTok{(}\DataTypeTok{opt_par_s=}\KeywordTok{c}\NormalTok{(b_}\DecValTok{5}\OperatorTok{$}\NormalTok{opt_c}\OperatorTok{$}\NormalTok{scale_e,b_}\DecValTok{10}\OperatorTok{$}\NormalTok{opt_c}\OperatorTok{$}\NormalTok{scale_e,b_}\DecValTok{20}\OperatorTok{$}\NormalTok{opt_c}\OperatorTok{$}\NormalTok{scale_e,b_}\DecValTok{40}\OperatorTok{$}\NormalTok{opt_c}\OperatorTok{$}\NormalTok{scale_e),}\DataTypeTok{N=}\KeywordTok{c}\NormalTok{(}\DecValTok{5}\NormalTok{,}\DecValTok{10}\NormalTok{,}\DecValTok{20}\NormalTok{,}\DecValTok{40}\NormalTok{))))}

\KeywordTok{ggplot}\NormalTok{()}\OperatorTok{+}
\StringTok{  }\KeywordTok{geom_line}\NormalTok{(}\DataTypeTok{data=}\NormalTok{param_n,}\KeywordTok{aes}\NormalTok{(N,opt_par_l,}\DataTypeTok{col=}\StringTok{"Location"}\NormalTok{))}\OperatorTok{+}
\StringTok{  }\KeywordTok{geom_line}\NormalTok{(}\DataTypeTok{data=}\NormalTok{param_n,}\KeywordTok{aes}\NormalTok{(N,opt_par_s,}\DataTypeTok{col=}\StringTok{"Scale"}\NormalTok{))}\OperatorTok{+}
\StringTok{  }\KeywordTok{geom_hline}\NormalTok{(}\DataTypeTok{yintercept =} \DecValTok{2}\NormalTok{, }\DataTypeTok{col=}\StringTok{"red"}\NormalTok{)}\OperatorTok{+}
\StringTok{  }\KeywordTok{geom_hline}\NormalTok{(}\DataTypeTok{yintercept =} \DecValTok{5}\NormalTok{, }\DataTypeTok{col=}\StringTok{"turquoise"}\NormalTok{)}\OperatorTok{+}
\StringTok{  }\KeywordTok{labs}\NormalTok{(}\DataTypeTok{y=}\StringTok{""}\NormalTok{, }
       \DataTypeTok{x=}\StringTok{"Number of Observations"}\NormalTok{, }
       \DataTypeTok{title=}\StringTok{"Location & Scale against N"}\NormalTok{)}
\end{Highlighting}
\end{Shaded}

\includegraphics{CWAKnit_files/figure-latex/unnamed-chunk-15-1.pdf}

\subsubsection{Stabilisation of Optimization
Paramaters}\label{stabilisation-of-optimization-paramaters}

The optimization paramaters stabilise as N increases. They approach
their respective true values as N increases.

\subsubsection{Similar Comparison for Squared Error and Dawid Sebastiani
Score}\label{similar-comparison-for-squared-error-and-dawid-sebastiani-score}

The squared error and dawid sebastiani scores will also stabalise as
using a larger N results in better estimation of mu and sigma. Better
estimates of mu and sigma result in better scores for squared error and
dawid sebastiani.

Also since, \[ {p_{F}=\mathrm{E}_{F}(z)} \] If the Brier Score
stabalises then so will squared error score.

\begin{Shaded}
\begin{Highlighting}[]
\CommentTok{# Attempted Task 10 to run 25 times and then average}


\CommentTok{# run = function (rn)\{}
\CommentTok{#   Num <- c(5,10,20,40)}
\CommentTok{#   }
\CommentTok{#   for (n in Num) \{}
\CommentTok{#     sum_n=0 }
\CommentTok{#     sum_l=0 }
\CommentTok{#     sum_s=0 }
\CommentTok{#     count=0}
\CommentTok{#     for (i in 1:rn) \{}
\CommentTok{#       c_n=rcauchy(n,location=2,scale=5)}
\CommentTok{#       b_n=brier_cauchy(c_n)}
\CommentTok{#       sum_n=sum_n+b_n$mean_d}
\CommentTok{#       sum_l=sum_l+b_n$opt_c$location_e}
\CommentTok{#       sum_s=sum_s+b_n$opt_c$scale_e}
\CommentTok{#       count=count+1}
\CommentTok{#   \}}
\CommentTok{#   print(c(sum_n/count,sum_l/count,sum_s/count))}
\CommentTok{#   \}}
\CommentTok{# \}}
\CommentTok{# run(25)}
\end{Highlighting}
\end{Shaded}


\end{document}
